\section{Languages and Frameworks}

    I have chosen to develop this as a web app as this is the most accessible form of application that can run on any web browser. This also fulfils the requirement of a desktop-scale application [section \ref{requirements-GUI}], maintaining support for variable aspect ratios.

    \subsection{TypeScript}

        \begin{wrapfigure}{r}{0.10\textwidth}
            \centering
            \includegraphics[width=0.10\textwidth]{ts-logo-512.png}
            %\caption{TypeScript logo \cite{typescript-logo}}
        \end{wrapfigure}

        TypeScript \cite{typescript} is a superset of the JavaScript language, the main difference being that TypeScript has the capabilities for static typing. I have chosen this as it combines the flexible and powerful components of native JavaScript with the clarity, efficiency and maintainability of statically typed languages. Hopefully this will allow me to write code quickly and cleanly, using my IDE's built-in TypeScript development tools.

        TypeScript compiles into plain JavaScript code and this is what will be running in the browser when the program is loaded.

    \subsection{JavaScript}

    \begin{wrapfigure}{r}{0.10\textwidth}
        \centering
        \includegraphics[width=0.10\textwidth]{ts-logo-512.png}
        %\caption{TypeScript logo \cite{typescript-logo}}
    \end{wrapfigure}

    \subsection{HTML/CSS}

        The HTML and CSS framework will handle most of the GUI visuals and layout. It is a very powerful tool and interacts with

\section{Network Builder}

    \subsection{User Interaction}

    \subsection{Graph Structure}

    \subsection{GUI}

\section{Network Testing}

    \subsection{Traffic Behaviour Modelling}

    \subsection{Agent Structure}

    \subsection{Maximum Graph Flow}

    \subsection{GUI}

    \subsection{Bezier Curvature}
