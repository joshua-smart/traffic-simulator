\section{Languages and Frameworks}

    I have chosen to develop this as a web app as this is the most accessible form of application that can run on any web browser. This also fulfils the requirement of a desktop-scale application [section \ref{requirements-GUI}], maintaining support for variable aspect ratios.

    \subsection{JavaScript}



    \subsection{TypeScript}

        \begin{wrapfigure}{r}{0.10\textwidth}
            \centering
            \includegraphics[width=0.10\textwidth]{ts-logo-512.png}
        \end{wrapfigure}

        TypeScript \cite{typescript} is a superset of the JavaScript language, the main difference being that TypeScript has the capabilities for static typing. I have chosen this as it combines the flexible and powerful components of native JavaScript with the clarity, efficiency and maintainability of statically typed languages. Hopefully this will allow me to write code quickly and cleanly, using my IDE's built-in TypeScript development tools.

        TypeScript compiles into plain JavaScript code and this is what will be running in the browser when the program is loaded.

    \subsection{HTML/CSS}

        The HTML and CSS framework will handle most of the GUI visuals and layout. This will interface directly with my JavaScript object code through DOM-manipulation statements. A few important ones are listed below:

        \begin{itemize}
            \item \begin{lstlisting}
document.querySelector(<string>);\end{lstlisting}
To retrieve a reference to the specified element in the document.
            \item \begin{lstlisting}
element.style.<property>;\end{lstlisting}
Allows access to an elements runtime CSS properties to be edited.
            \item \begin{lstlisting}
document.createElement(<tag>);\end{lstlisting}
Creates an element with a specified HTML tag that can be subsequently added to the document.
            \item \begin{lstlisting}
element.appendChild(<childElement>);\end{lstlisting}
Adds the child element as a sub-element of the parent.
            \item \begin{lstlisting}
element.innerHTML;\end{lstlisting}
Accesses the raw HTML contained within the element as a string to be changed or reassigned.
            \item \begin{lstlisting}
element.addEventListener(<listener>);\end{lstlisting}
This will be a vital part of my program, allowing code to respond to document events such as clicks, keypresses and scrolls.
        \end{itemize}

\section{Network Builder}

    \subsection{User Interaction}

    \subsection{Graph Structure}

    \subsection{GUI}

\section{Network Testing}

    \subsection{Traffic Behaviour Modelling}

    \subsection{Agent Structure}

    \subsection{Maximum Graph Flow}

    \subsection{GUI}

    \subsection{Bezier Curvature}
