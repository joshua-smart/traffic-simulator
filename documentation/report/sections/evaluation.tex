\section{Summary}

    I have learned many new skills and techniques during this project such as unit testing and program architectures. I think the main success I had with this project was my own approach to development; this approach mainly focused on planning and designing the program before development began. This process is documented in \autoref{design}, I spent a decent chunk of time writing my ideas and producing diagrams, when the time came to begin development this wealth of resources drastically sped up my pace of development as I could devote my attention to specific implementations instead of the program design.

    Along with this I discovered and researched the Model-View-Controller architecture which I then utilised in my own program. This helped me greatly as it allowed me to compartmentalise the different sections and create features separately that I could then bring together to form a cohesive program. I would like to have tried a different architecture such as Hex (Ports \& Adapters) to develop my program, but time constraints as well as ignorance to these other options led me to continue on with the aforementioned design.

    During the development process I hit an unknown bug. In short my implementation of Dijkstra's algorithm would not correctly detect when a path between two vertices did not exist, causing the agents to take invalid paths across the network. I had spent a while trying to locate the source of this issue when I decided to implement unit tests. Writing a suite of unit tests to validate the functionality of the graph class allowed me to quickly pinpoint the source of this error as well as spot a few potential future problems in other parts of the program.

\section{Third Party Feedback}

    For this section I requested one of my peers to use and make comments on my program, below is a transcript of their response:

    \begin{quote}
        Visually, the program is lovely, soft colours and clear buttons that are positioned neatly on the screen to be out of the way but clearly present. Design is powerful but not overwhelming, customisation options are massive, with panning and zoom allowing for complex and large simulations yet none seems overbearing and all is intuitive and feels comfortable to use. Overall, this is a wonderful program that is clean, highly functional and yet incredibly intuitive.
    \end{quote}

    These generous comments highlight a few of the main objectives I intended to hit in this project. Namely the design of the user interface, which my client described as "powerful but not overwhelming". This mirrors my own objective listed in \autoref{analysis:objectives}. Similarly the client talked about an "incredibly intuitive [program]", highlighting my intention to make an easy-to-use interface, and it appears that my focus on this in researching existing solutions has paid off.

\section{Objective Evaluation}

    In this section I will go through each objective listed in \autoref{analysis:objectives} and evaluate to what extent I have achieved this in my technical solution.

    \subsection{Road Network Rendering}

        \begin{quote}
            \textbf{The program should be capable of procedurally drawing a road network to the UI using point vertices and cubic Bezier curves to model roads.}
        \end{quote}

        As demonstrated in \autoref{testing:t1}, the program is fully capable of rendering a road network structure to the UI. This forms the basis of the user interaction allowing them to visualise the underlying graph data structure. The road visualisation also gives the user an obvious impression of direction with a small direction arrow at the midpoint of the edge.

    \subsection{Vertex Limit}

        \begin{quote}
            \textbf{The number of vertices should be limited to 50 to prevent slowing down the simulation.}
        \end{quote}

        I have successfully implemented a limit of 50 vertices to the road network, this prevents the route finding algorithm from becoming a performance issue. As seen in the video tests, there is no noticeable lag spike when agents are spawned.

    \subsection{Save \& Load}

        \begin{quote}
            \textbf{The user should be able to save and load road networks in and out of the program.}
        \end{quote}

        This I believe is an essential feature for a program of this kind. Allowing the user to import and export projects, I have thoroughly tested this functionality and it all gives expected behaviour. I implemented this using JavaScript's native JSON processing and a package to interface with the browser's file save API. Loading the files followed a similar procedure, but in reverse. This was a useful exercise in file handling

    \subsection{Road Network Builder}

        \begin{quote}
            \textbf{The program should support a click-and-drag based interaction, allowing the creation of vertices and connections and further editing of the network.}
        \end{quote}

        This feature was inspired by the control scheme of mini-motorways, as I found it to be intuitive and easy to memorise. My implementation of this uses the left mouse button as well as the shift key to modify the action, allowing many different actions to be performed with few controls.

    \subsection{Undo \& Redo}

        \begin{quote}
            \textbf{The user should also be able to undo and redo actions to the road network.}
        \end{quote}

        Each edit the user makes to the graph is logged as a user action and they are able to navigate through the past states of the graph with the undo feature. Similarly, the user is able to restore actions that have been undone with redo. This is an essential feature for this category of program and makes the developement of a netwrok by the user a much friendlier experience.

    \subsection{Agent Simulation}

        \begin{quote}
            \textbf{The program should be able to simulate cars driving sensibly across the road network.}
        \end{quote}

        Implemented in the agent class I have created an algorithm that allows each agent to asses the position and movement of the other agents in the network and calculate how much it should accelerate or decelerate in a given frame. Applying this algorithm at each timestep and updating the positions accordingly produces and emergent behaviour of the traffic: cars will give way to each other when merging at a junction, cars will maintain a sensible distance from the vehicle ahead, and they will also accelerate up to but not exceed a set speed limit.

    \subsection{Agent Limit}

        \begin{quote}
            \textbf{The maximum number of agents active at any time should be limited to 60.}
        \end{quote}

        This was a necessary limitation as the complexity of the acceleration algorithm scales with the square of the number of agents in the network, a maximum agent count of 60 ensures that this does not affect the framerate of the simulation.

    \subsection{Data Output}

        \begin{quote}
            \textbf{Relevant output data described in \autoref{analysis} should be outputted to both the UI and downloadable files.}
        \end{quote}

        As shown in the testing section, the screen displays a summary of the current simulation statistics as it is running. Similarly, at any point in the simulation a complete table of data over time can be downloaded by using the buttons at the bottom of the summary GUI.

    \subsection{Interface Design}

        \begin{quote}
            \textbf{There should be a minimalist interface with the majority of screen real-estate allocated to the network visualisation.}
        \end{quote}

        Also visible in the testing section are many instances of the GUI design, this shows a top toolbar containing the controls for the simulation, and a large canvas containing the road network visualisation. Providing a large interface for interacting with it.

    \subsection{Desktop Compatibility}

        \begin{quote}
            \textbf{This program should support desktop size displays.}
        \end{quote}

        As shown in many testing images, this program supports a desktop browser (most were recorded using firefox with a 1920x1080 display). This allows plenty of space for the elements to be displayed on the screen. This program is also intended to be used most commonly on this kind of platform.

    \subsection{UI Scaling}

        \begin{quote}
            \textbf{It should also be compatible with variable aspect ratios, resizing UI elements to fit the screen.}
        \end{quote}

        In the final entry of my testing section, I used the developement tools provided by my browser to test the compatibility with many different sizes of window. The elements in the toolbar are capable of sliding laterally to fit the available space and the main canvas will stretch or compress to fill the rest of the window.

\section{Technical Evaluation}

    In this section I will review my codebase and pick out specific implementations or design choices that could be improved upon. I will explore where and why these decisions were made as well as providing improvements.

    \subsection{Pathfinding Algorithms}

        As described in \autoref{design:pathfinding}, in my code I implemented Dijkstra's algorithm to facilitate finding routes across the road network. This has a simple implementation and is a fairly well known path finding algorithm. On reflection, I could have instead implemented the A* search algorithm as this uses a heuristic technique which is on average more efficient than Dijkstra's.

    \subsection{Bezier lookup table}

        \autoref{design:linear-interpolation-bezier} shows the algorithm I wrote for interpolating a Bezier curve by distance using a lookup table of sampled points. Another further improvement I could have made is the use of linear interpolation between the data points in order to improve the accuracy of the approximation.

        With the current implementation I look for the first distance sample $\geq$ the target and take it's value of $t$. Alternatively, by finding the last distance sample below the target ($d_1$, $t_1$) and the first distance sample above the target ($d_2$, $t_2$), I could interpolate to get:

        \[t = t_1 + (t_2 - t_1) * \frac{d - d_1}{d_2 - d_1}\]

        Providing a better approximation of the curve, although this would have required some additional consideration into retrieving these values.

    \subsection{Control Structure}

        Documented in \autoref{subsection:user-interaction}, I utilised a state machine to manage all the controls of the program. This was a very useful model for implementing the network builder controls as it allowed me to rely on state transitions to remember which point of the action a user is currently performing. However this did have some unforeseen side effects for the rest of the program.

        My program implements \mintTS{simulationActive} as it's own state, and this state can only be transitioned out of my either stopping the simulation (transitioning back to the idle state) or pausing the simulation (at which point the program enters the \mintTS{simulationPaused} state). This causes the user to be unable to save the simulation while it is running or even while it is paused, as this event branches off from the idle state.

        On reflection, it would have improved the system to isolate the state machine responsibility to just the network builder, using more traditional methods of control flow to manage the main framework of the application.

\section{Conclusion}

    Overall I have enjoyed this project and I am pleased with the solutions ability to meet the objectives I had set. In order to develop this program further I would add support for vertical height differences between road sections (bridges and underpasses), as well as more road features such as parking lots or traffic lights. My proudest achievement within this process was my focus on program design and architecture, designing what I believe is quite an elegant solution to the problem.
